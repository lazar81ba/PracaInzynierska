\chapter{Podsumowanie}
Celem niniejszej pracy inżynierskiej było zaprojektowanie oraz zaimplementowanie kompletnej aplikacji, umożliwiającej komunikację oraz wymianę informacji między autorami projektów. Proces implementacji odbywał się zgodnie z metodyką zwinną oraz zastosowano w nim kilka wybranych technik wytwarzania oprogramowania stosowanych w obrębie wyżej wymienionej metodyki. Jednym z elementów aplikacji jest system rekomendacji, którego zadaniem było polecanie projektów dostosowanych do zainteresowań użytkownika. Jednocześnie miał on za zadanie filtrować projekty nie związane z branżą danego użytkownika. Do jego implementacji wykorzystano algorytm Content-based filtering dla reprezentacji binarnej. Ponadto przy projektowaniu i implementacji skupiono się na wygodnym i przejrzystym interfejsie użytkownika. Finalnie dostarczono produkt o minimalnej wartości biznesowej zgodny z ustalonymi wymaganiami.

\section{Perspektywy rozwoju}

Na implementację aplikacji został przeznaczony określony czas, w którym należało spełnić konkretne wymagania. Wymóg ten sprawił, że dostarczenie wszystkich funkcjonalności określonych przez klienta stało się niemożliwe przy obecnej zdolności. W tej sytuacji konieczne było ustalenie najważniejszych \textit{User stories}, które mogły zostać zaimplementowane w danym czasie. Za pomocą techniki \textit{Planning poker} ustalono koszt wyprodukowania każdej z nich, co pozwoliło na określenie, które z nich mają zostać wykonane. W ramach dalszego rozwoju aplikacji należałoby skupić się na dostarczeniu funkcjonalności, które nie były możliwe do spełnienia tzn. komunikator, powiadomienia email, wysyłanie zaproszeń do użytkowników. Przy ustaleniu planu na kolejną iterację należy również pamiętać o możliwości pojawienia się nowych wymagań użytkownika takich jak np. możliwość rejestrowania oraz edycji użytkowników, panel administratora. Historyjki te należy ponownie ocenić za pomocą \textit{Planning Poker} oraz ustalić wartość biznesową każdej z nich.

\section{Ocena systemu rekomendacji}

Obecna wersja zaimplementowanego systemu rekomendacji dobrze spełnia zadania określone we wstępnych wymaganiach. Pozwala on na wyświetlenie użytkownikowi tylko tych projektów, które są dla niego potencjalnie interesujące. Jednakże na tym etapie występuje kilka problemów wymagających naprawy bądź ulepszenia. Pierwszym z nich jest problem zimnego startu. Pomimo faktu, że system opiera się na zawartości przez co nie potrzebuje ogromnej ilości danych do stworzenia wystarczająco dobrej rekomendacji, wciąż pojawia się problem zimnego startu. Nowo utworzony użytkownik musi zaobserwować jakikolwiek projekt, aby rekomendacja mogła zostać stworzona. Rozwiązaniem tego problemu może być przerobienie algorytmu Content-based na algorytm hybrydowy. W przypadku, gdy użytkownik nie zaobserwowałby żadnego projektu predykcja mogłaby zostać osiągnięta za pomocą danych takich jak kraj, uczelnia, specjalizacja. Hybrydowy system rekomendacji mógłby zastosować algorytm Collaborative filtering np. do polecenia projektów osób z taką samą specjalizacją jak użytkownik docelowy. Innym problemem który występuje jest wzrost złożoności obliczeniowej proporcjonalny do ilości nowych danych. Content-based filtering tworzy predykcję na podstawie wszystkich rozpatrywanych elementów, przez co im większa ilość danych tym większa liczba obliczeń. Problem ten może zostać naprawiony poprzez odpowiednią optymalizację algorytmu, zastosowanie innych struktur danych oraz wykonanie obliczeń na chmurze obliczeniowej. Dodatkową możliwością ulepszenia działania algorytmu jest wprowadzenie oznaczenia projektów jako nie interesujące. Takie oznaczenie spowodowałoby większą różnorodność oraz lepsze dopasowanie profilu użytkownika, jednak do określenia dobrej predykcji wymagałoby większego zbioru danych.