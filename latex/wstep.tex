\chapter{Wstęp}
\label{cha:wstep}

\section{Cel pracy}
\label{sec:celPracy}

Celem pracy jest zaprojektowanie aplikacji internetowej o nazwie \mbox{\textit{ProjectSHARE}}, która ma na celu wspomagać oraz uprościć wymianę informacji między zespołami zaangażowanymi w tworzenie
projektów o podobnej tematyce. Aplikacja ma postać platformy, dzięki której użytkownik
może: stworzyć projekt prywatny bądź publiczny, wyszukać inicjatywy tworzone przez
innych użytkowników oraz dołączyć do istniejącego już projektu. Dodatkowo platforma zawiera system rekomendacji projektów uwzględniający preferencje użytkownika.

\section{Opis problemu}
\label{sec:opisProblemu}
Tworząc różnego rodzaju projekty takie jak np. praca inżynierska, projekt uczelniany, 
projekt tworzony dla koła naukowego czy zaawansowany produkt biznesowy, często pojawiają się
problemy, które blokują dalszą pracę. Twórcy sięgają wtedy po źródła dostępne w
internecie, jednak nie zawsze są one w stanie im pomóc bądź wiarygodność ich autorów jest
nieznana. W danej sytuacji jednym z najbardziej pomocnych rozwiązań jest kontakt z osobą
która ma większe doświadczenie oraz miała styczność z implementacją podobnej
funkcjonalności w celu konsultacji problemu. Możliwość takiego kontaktu zapewnia kilka
popularnych repozytorium takich jak np. GitHub, jednak mają one kilka wad: zawierają głównie
projekty z branży IT, brakuje możliwości łatwego wyszukania projektów oraz często
wymagają udostępnienia całej implementacji, ponieważ stworzenie prywatnego
repozytorium wymaga opłaty. Można również dodać, że firmy produkujące oprogramowanie
biznesowe mają swoje własne repozytoria, więc wyszukanie projektu oraz kontakt z
doświadczonym pracownikiem jest trudny. 


Po przeanalizowaniu wyżej wymienionych scenariuszy powstał pomysł na stworzenie aplikacji         
\mbox{ProjectSHARE}, która będzie uwzględniać każdy z nich w fazie projektowania oraz rozwiąże wynikające z nich problemy użytkownika systemu skierowanego do autorów projektów.

\newpage

\section{Zakres pracy}
\label{sec:zakresPracy}

W ramach pracy inżynierskiej wykonano:
\begin{itemize}
	\item Przegląd literatury dotyczącej metodologii wytwarzania oprogramowania oraz stosowanych w ich obrębie technik
	\item Przegląd literatury dotyczącej technologii potrzebnych do stworzenia aplikacji internetowej oraz zasady działania każdej z nich
	\item Przegląd literatury dotyczącej algorytmów stosowanych w systemach rekomendacji oraz wybór odpowiedniego algorytmu bazując na ustalonych założeniach systemu	
	\item Projekt architektury aplikacji, interfejsu użytkownika oraz modelu bazy danych
	\item Implementację elementów tworzonej platformy określonych w wymaganiach
	\item Implementację wybranego algorytmu uczenia maszynowego dotyczącego systemu rekomendacji

	
\end{itemize}